\documentclass[a4paper,14pt,russian]{extreport}
 
\usepackage{extsizes}
\usepackage{cmap} % для кодировки шрифтов в pdf
\usepackage[T2A]{fontenc}
\usepackage[utf8]{inputenc}
\usepackage[russian]{babel}
%\usepackage{pscyr} % набор красивых шрифтов
 
\usepackage{graphicx} % для вставки картинок
\usepackage{amssymb,amsfonts,amsmath,amsthm} % математические дополнения от АМС
\usepackage{indentfirst} % отделять первую строку раздела абзацным отступом тоже
\usepackage[usenames,dvipsnames]{color} % названия цветов
\usepackage{makecell}
\usepackage{multirow} % улучшенное форматирование таблиц
\usepackage{ulem} % подчеркивания
 
\linespread{1.3} % полуторный интервал
\renewcommand{\rmdefault}{ftm} % Times New Roman
\frenchspacing

\usepackage{fancyhdr}
\pagestyle{fancy}
\fancyhf{}
\fancyhead[R]{\thepage}
\fancyheadoffset{0mm}
\fancyfootoffset{0mm}
\setlength{\headheight}{17pt}
\renewcommand{\headrulewidth}{0pt}
\renewcommand{\footrulewidth}{0pt}
\fancypagestyle{plain}{ 
    \fancyhf{}
    \rhead{\thepage}}
\setcounter{page}{1} % начать нумерацию страниц с №1

\usepackage[tableposition=top]{caption}
\usepackage{subcaption}
\DeclareCaptionLabelFormat{gostfigure}{Рисунок #2}
\DeclareCaptionLabelFormat{gosttable}{Таблица #2}
\DeclareCaptionLabelSeparator{gost}{~---~}
\captionsetup{labelsep=gost}
\captionsetup[figure]{labelformat=gostfigure}
\captionsetup[table]{labelformat=gosttable}
\renewcommand{\thesubfigure}{\asbuk{subfigure}}

\usepackage{titlesec}
\titleformat{\chapter}[display]
    {\filcenter}
    {\MakeUppercase{\chaptertitlename} \thechapter}
    {8pt}
    {\bfseries}{}
\titleformat{\section}
    {\normalsize\bfseries}
    {\thesection}
    {1em}{}
\titleformat{\subsection}
    {\normalsize\bfseries}
    {\thesubsection}
    {1em}{}
     
% Настройка вертикальных и горизонтальных отступов
\titlespacing*{\chapter}{0pt}{-30pt}{8pt}
\titlespacing*{\section}{\parindent}{*4}{*4}
\titlespacing*{\subsection}{\parindent}{*4}{*4}

\usepackage{geometry}
\geometry{left=3cm}
\geometry{right=1.5cm}
\geometry{top=2.4cm}
\geometry{bottom=2.4cm}

\usepackage{enumitem}
\makeatletter
    \AddEnumerateCounter{\asbuk}{\@asbuk}{м)}
\makeatother
\setlist{nolistsep}
\renewcommand{\labelitemi}{-}
\renewcommand{\labelenumi}{\asbuk{enumi})}
\renewcommand{\labelenumii}{\arabic{enumii})}

\usepackage{tocloft}
\renewcommand{\cfttoctitlefont}{\hspace{0.38\textwidth} \bfseries\MakeUppercase}
\renewcommand{\cftbeforetoctitleskip}{-1em}
\renewcommand{\cftaftertoctitle}{\mbox{}\hfill \\ \mbox{}\hfill{\footnotesize Стр.}\vspace{-2.5em}}
\renewcommand{\cftchapfont}{\normalsize\bfseries \MakeUppercase{\chaptername} }
\renewcommand{\cftsecfont}{\hspace{31pt}}
\renewcommand{\cftsubsecfont}{\hspace{11pt}}
\renewcommand{\cftbeforechapskip}{1em}
\renewcommand{\cftparskip}{-1mm}
\renewcommand{\cftdotsep}{1}
\setcounter{tocdepth}{2} % задать глубину оглавления — до subsection включительно

\newcommand{\empline}{\mbox{}\newline}
\newcommand{\likechapterheading}[1]{ 
    \begin{center}
    \textbf{\MakeUppercase{#1}}
    \end{center}
    \empline}
    
\makeatletter
    \renewcommand{\@dotsep}{2}
    \newcommand{\l@likechapter}[2]{{\bfseries\@dottedtocline{0}{0pt}{0pt}{#1}{#2}}}
\makeatother
\newcommand{\likechapter}[1]{    
    \likechapterheading{#1}    
    \addcontentsline{toc}{likechapter}{\MakeUppercase{#1}}}

\usepackage[square,numbers,sort&compress]{natbib}
\renewcommand{\bibnumfmt}[1]{#1.\hfill} % нумерация источников в самом списке — через точку
\renewcommand{\bibsection}{\likechapter{Список литературы}} % заголовок специального раздела
\setlength{\bibsep}{0pt}

\parindent = 1cm

\begin{document}
Библиотека cuSolver основана на более ранних библиотеках cuBlas и cuSparse. Она соединяет в одну три непересекающиеся библиотеки, каждая из которых может использоваться независимо от других или совместно с другими библиотеками в комплекте. 
\par
Цель создания cuSolver - предоставить возможности пакета LAPACK, такие как разложение матрицы на множители и др. (triangular solve routines) для переопределённых матриц, солверы метода наименьших квадратов и характеристического числа для разрежённых матриц(большинство нулей). 
\par
Первая часть cuSolver называется cuSolverDN (Dense LAPACK) и используется для факторизации плотных матриц  и обработки подпрограмм таких, как LU (факторизация матрицы $A$ вида $A = LU$, где $L$ - нижняя треугольная матрица, $U$ - верхняя треугольная), QR (факторизация матрицы $A$ вида $A = QR$, где $Q$ - ортогональная матрица, $R$ - верхняя треугольная), SVD (сингулярное разложение) и LDLT.
\par
Следующая часть cuSolverSP: Sparse LAPACK. Она была разработана длоярешения резрежённой системе линейных ураавнений $Ax=b$, а также решению задачи наименьших квадратов $x = $ argmin $||A*z-b||$, где разрежённая матрица $A \in R_(m\times n)$, $b \in R_m$ и $x\in R_n$. В основе лежит алгоритм QR-факторизации. Матрица $A$ принимается в формате CSR (compressed sparse row). 
\par
CSR формат представляет матрицу в виде трёх массивов: первый содержит ненулевые элементы, второй номера строк и третий - столбцов. 
\par
И последняя cuSolverRF: Refactorization
\par
LAPACK (Linear Algebra PACKage) - библиотека с открытым исходным кодом, содержащая методы для решения основных задач линейной алгебры. Написана на языке Fortran с использованием другой библиотеки BLAS и является развитием пакета LINPACK.
BLAS (Basic Linear Algebra Subprograms) - интерфейс, базовые подпрограммы линейной алгебры:
\begin{enumerate}
\item Векторные операции вида $y\Leftarrow\alpha x + y$ - операции скалярного произведения, взятия нормы вектора и др.
\item Операции матрица-вектор вида $y\Leftarrow\alpha Ax + \beta y$, решение $Tx = y$ для $x$ с треугольной матрицей $T$ и др.
\item Операции матрица-матрица вида $C\Leftarrow \alpha AB + \beta C$, решение $B\Leftarrow \alpha T^(-1)B$ для треугольной матрицы $T$ и др.
\end{enumerate}
LINPACK — программная библиотека, написанная на языке Фортран, которая содержит набор подпрограмм для анализа и решения плотных систем линейных алгебраических уравнений. Активно исопльзует интерфейс BLACK.
\par
cublas<t>gemm() функция, которая выполняет произведение двух матриц. cublasGemmEx() может специализировать stream.
\par


\end{document}
\grid
\grid
